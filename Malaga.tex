\documentclass[a4paper]{article}
\usepackage[printonlyused]{acronym}
\usepackage{siunitx}
\usepackage{graphicx}
\usepackage{pgfpages}
\usepackage{subcaption}
\usepackage{varioref}
\usepackage{url}
\usepackage{fontspec}
\usepackage{polyglossia}
\usepackage{amsthm}
\usepackage{fancyhdr}% http://ctan.org/pkg/fancyhdr
\usepackage{geometry}
\newcommand{\as}{\\[14pt]}
\newcommand{\s}{\\[7pt]}
\newcommand{\is}{\\[2pt]}
\newcommand{\no}{\noindent}
\newcommand{\ka}{\hspace*{0.5cm}}
\newcommand{\ma}{\hspace*{1cm}}
\newcommand{\ga}{\hspace*{1.5cm}}
\newcommand{\li}{\left|}
\newcommand{\re}{\right|}
\newcommand{\const}{\text{const.}}
\newcommand{\z}{\text}
\newcommand{\terminal}[1]{\colorbox{black}{\textcolor{white}{{\fontfamily{phv}\selectfont \scriptsize{#1}}}}}
\newcommand{\plugin}[1]{\textit{\flq#1\frq}}
\newcommand{\ra}{$\rightarrow$ }
\definecolor{cadmiumgreen}{rgb}{0.0, 0.42, 0.24}
\newcommand{\itemfill}{\setlength{\itemsep}{\fill}}
\newcommand{\orderof}[1]{$\mathcal{O}\left(#1\right)$}
\newcommand{\fig}[3][h!]{\begin{figure}[#1]\centering\includegraphics[width={#3}\textwidth]{#2}\end{figure}}
\newcommand{\figr}[2]{\begin{figure}\centering\includegraphics[width={#2}\textheight, angle=-90]{#1}\end{figure}}
\newcommand{\subfiga}[3][0.45]{\begin{subfigure}{{#1}\textwidth}\centering\includegraphics[height={#3}\textheight]{#2}\end{subfigure}}
\newcommand{\subfigar}[3][0.45]{\begin{subfigure}{{#1}\textwidth}\centering\includegraphics[width={#3}\textheight, angle=-90]{#2}\end{subfigure}}

\geometry{verbose,a4paper,tmargin=25mm,bmargin=25mm,lmargin=20mm,rmargin=20mm}
\setcounter{secnumdepth}{2}
\newfontfamily{\cyrillicfont}{Carlito}  %{Liberation Mono}
\setmainlanguage{russian} 
\setotherlanguage{english}
\pagestyle{fancy}% Change page style to fancy
\fancyhf{}% Clear header/footer
\fancyhead[L]{\leftmark\hfill Отпуск в Малаге}
% \fancyhead[R]{\rightmark}
\graphicspath{ {pics/} }
% \renewcommand\subsectionmark[1]{\markleft{\thesubsection~#1}} % mit Nummer 

\makeindex
\begin{document}

\section{первый $-$ ЧЕТВЕРГ (четырнадцатое декабря)}
Месяц тому назад мы решили провести наш отпуск в Малаге. Я прилетела в аэропорт первой, через час прилетел Михаэль. Вместе мы забрали нашу машину в службе проката и поехали в город.
\fig{0}{.9}
\begin{figure}[h]\centering
	\subfiga{car}{.2}
	\subfiga{car1}{.2}
\end{figure}
Наша квартира была очень близко к центру, мы оставили наши вещи, и пошли искать ресторан. Я нашла несколько мест на Tripadvisor, и мы отправились к одному из них. Это был ресторанчик на рынке, где мы ели креветки, рыбу и жареную икру. После обеда у нас была запланирована экскурсия в крепость 11 века Alcazaba. Эта крепость очень старая, но хорошо сохранилась. Там было много фонтанов и цветов.\newpage
\fig{1}{.8}
\begin{figure}[h]\centering
	\subfiga[.54]{al1}{.22}
	\subfigar[.36]{al2}{.22}
\end{figure}\noindent
Мы приехали в Малагу перед Рождеством, город был очень красиво украшен и мы пошли смотреть шоу в центре. На главной площади было много людей, играла громкая музыка, все пели «Feliz Navidad». Мы немного погуляли по главной улице и пошли пить Tinto de Verano в Casa Lolla. По дороге домой мы купили продукты и забрали нашу машину с паркинга.\\
\fig[h!]{fn1}{.5}\newpage

\section{второй - ПЯТНИЦА (пятнадцатое декабря)}
В этот день у Михаэля был День Рождения! Утром я подарила ему подарок, и мы поехали к Caminito del Rey. Это очень крутое место! Мы ходили по тропе вдоль ущелья. Было немножко страшно, но очень красиво и интересно.
\fig{cdr0}{.7}
\begin{figure}[h!]\centering
	\subfiga{cdr1}{.2}
	\subfiga{cdr2}{.2}
\end{figure}\\
После тропы мы решили заехать в город Ronda. По дороге мы увидели озеро, оставили машину и пошли к нему ближе, чтобы полюбоваться.
\begin{figure}[h!]\centering
	\subfiga{lake1}{.2}
	\subfiga{lake2}{.2}
\end{figure}\newpage\noindent
В Ронде мы гуляли по старой части города и рассматривали огромный мост. Этот город находится на скале, и некоторые дома построены над обрывом. Мы сделали очень много красивых фото за этот день. Вечером мы очень устали и пошли пить шоколад с чуррос, купили сувениры и хамон иберико. Пора ехать обратно в Малагу.
\begin{figure}[h!]\centering
	\subfiga{ro1}{.2}
	\subfiga{ro2}{.2}
\end{figure}\\
Вечером в Малаге очень трудно найти место для парковки. Мы нашли место только после 20 (двадцати) минут поисков. На ужин мы ели тапас и пили Tinto de Verano. Я украсила тапас Михаэля свечами, и он загадал желание. Какой чудесный день!

\section{третий - СУББОТА (шестнадцатое декабря)}
На завтрак мы ели хамон, который купили в Ронде. В 11 (одиннадцать) часов утра у нас была экскурсия по городу. Когда у нас был перерыв на отдых, Михаэль купил себе шарф и сладости для нас. После экскурсии мы ели паэлью с морепродуктами. Это так вкусно! 
\begin{figure}[h!]\centering
	\subfiga{fw1}{.2}
	\subfiga{fw2}{.2}
\end{figure}\\
Вторую половину дня мы провели в «белой» деревне на востоке от Малаги, Фрихилиане. В деревне было очень тихо и спокойно, без туристов. Мы увидели, как растет авокадо и грейпфрут (взяли несколько плодов с собой:) Взобрались на вершину холма и смотрели закат солнца.
\fig{fri1}{1}\newpage
\begin{figure}[h!]\centering
	\subfiga{fri2}{.2}
	\subfiga{fri3}{.2}
\end{figure}\noindent
По дороге в Малагу у нас был необычный ужин у моря. Мы если рыбу, которую испанцы жарят на костре в лодке. Также в этом ресторане мы попробовали пиво Аlhambra. Это был очень вкусный ужин!

\section{четвертый - ВОСКРЕСЕНЬЕ (семнадцатое декабря)}
Утром мы пошли кушать чуррос в самое старое кафе в Малаге, Аranda. Мы заказали 8! (восемь) чуррос и горячий шоколад. Это очень большое кафе - они могут накормить своими чуррос всю Малагу. Сделав бутерброды, мы поехали в парк с животными La Paloma. В этом парке свободно гуляют петухи и курицы, бегают кролики, летают попугаи, плавают гуси с утками. Парк нам очень понравился! Мы ели там наши бутерброды и кормили птиц с рук. После парка мы поехали смотреть замок Colomares. Этот замок посвящен памяти Колубма. Замок был достроен в 1994 (тысяча девятьсот девяносто четвертом году). Этот замок не похож на другие: он маленький, но в нем много архитектурных стилей и истории Испании. Это такое романтическое место! 
Недалеко от замка находится еще одна «белая» деревня в горах, Михас. Туда мы и поехали! Мы увидели небольшой концерт испанских песен, осликов, безумные елки из пластиковых бутылок и красивую панорамную смотровую площадку. Нужно было спешить обратно – мы планировали еще посетить музей Пикассо в Малаге. Возле музея была небольшая очередь, потому что в 16:00 вход в музей был бесплатным. Нам понравились картины и скульптуры в этом музее. Довольные мы пошли пить вино москатель в ресторан El Pimpi. 

\section{пятый - ПОНЕДЕЛЬНИК (восемнадцатое декабря)}
Рано утром мы поехали в пещеры недалеко от городка Нерха. Эти пещеры нас очень впечатлили - они огромного размера! Там было много сталактитов и сталагмитов. Я купила себе на память голубой камень в сувенирном магазине, и мы поехали искать секретный пляж. Пляж оказался нудистским. Вода была очень чистой, но холодной. Несмотря на это Михаэль немного поплавал. Вернувшись в Малагу мы пошли на рынок, чтобы купить экзотические фрукты. Больше всего нам понравился фрукт черимойя. На вкус он немного напоминает грушу.  В этот день мы осмотрели главный собор Малаги. У этого собора достроена только одна башня, поэтому местные называют собор La Manquita. После собора мы пошли к Хибральфаро, второй части крепости. У нас оставался только час времени на осмотр этого места. С этой крепости город виден как на ладони. Начинался закат, небо было залито розовой краской, город покрывал теплый свет заходящего солнца. Немножко грустно, что это был конец нашей поездки. Потом мы посмотрели порт и набережную. Мы даже успели посмотреть Рождественское шоу еще раз. На ужин мы ели вкусные тапас в Casa Lolla и пили вино в El Pimpi.

\section{шестой - ВТОРНИК (девятнадцатое декабря)}

Это был наш последний день в Малаге. Рано утром Михаель отвез меня в аэропорт, а сам поехал в Торемолинос. Дома я любовалась нашими фотографиями. Вечером мы разговаривали с Михаэлем в скайпе как обычно. Нам очень понравилась эта поездка! Я жду нашу новую встречу!


\end{document} 
